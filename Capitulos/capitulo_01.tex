%---------------------------------------------------------------------
%
%                          Capítulo 1
%
%---------------------------------------------------------------------

\chapter{Elementos básicos de la mecánica clásica y cuántica}

\begin{FraseCelebre}
\begin{Frase}
    Si consigo ver más lejos es porque\\
    he conseguido auparme a\\
    hombros de gigantes.
\end{Frase}
\begin{Fuente}
    Isaac Newton
\end{Fuente}
\end{FraseCelebre}



%-------------------------------------------------------------------
\section{introducción}
%-------------------------------------------------------------------
\label{cap1:sec:Introduccion}

Toda teoría que desea describir el comportamiento de las partículas elementales se necesita desarrollar bajo  el marco de la teoría cuántica de campos (\textbf{TCC} o en sus siglas en inglés \textbf{TQF}). Aquí se pueden encontrar teorías como la electrodinámica cuántica, el modelo estándar, la teoría de cuerdas, etc.\
La teoría cuántica de campos es el pilar fundamental de nuestra descripción de la Naturaleza. Juega un papel clave en física de la materia condensada,  mecánica estadística y por su puesto en física de altas energías. La teoría cuántica de campos es una potente herramienta matemática que nos permite describir a nuestro universo a distancias $ 10^{-10} \sim 10^{-20} m$, pero la pregunta que emerge de todo esto es ¿Qué es un campo cuántico? y ¿Porqué se necesita un campo cuántico para describir a las partículas que observamos en la naturaleza?\\
De modo que necesitamos una forma de tener una formulación de la mecánica cuántica junto con la relatividad especial, ya que las partículas elementales que conforman a la naturaleza poseen estas características.

%-------------------------------------------------------------------
\section{Convenciones}
%-------------------------------------------------------------------
\label{cap1:sec:Convenciones}

Antes de empezar a profundizar en los cálculos, es necesario definir unas reglas o convenciones como pueden ser las unidades a utilizar. En teorías que describen partículas fundamentales es común utilizar las llamadas \textit{unidades naturales}. 
El uso de este sistema de unidades trae consigo varias ventajas, una de ellas es que simplifican mucho la estructura de las ecuaciones porque elimina las constantes de proporcionalidad y hace que los resultados no dependan del valor de las constantes y se pueda enfocar principalmente en la física del problema.
El sistema mide varias de las magnitudes fundamentales del universo: \textit{tiempo, longitud, masa, carga eléctrica y temperatura}. Se definen haciendo que las cinco constantes de la tabla (\ref{tabla1:Unidades naturales}) tomen el valor la unidad cuando se expresen en ecuaciones y cálculos en dicho sistema.

\begin{table}[h]
\centering    
    \begin{tabular}{| c | c |}
        \hline
        Constante & Símbolo \\
        \hline
        Velocidad de la luz & $c$ \\
        \hline
        Constante de gravitación universal  & $G$ \\
        \hline
        Contante reducida de Plank & $\hbar$ \\
        \hline
        Constante de fuerza de Coulomb & $\frac{1}{4\pi\epsilon_0}$ \\
        \hline
        Constante de Boltzmann & $k$ \\
        \hline
    \end{tabular}
\caption{Unidades naturales.}
\label{tabla1:Unidades naturales}
\end{table}



%-------------------------------------------------------------------
\section{Mecánica clásica}
%-------------------------------------------------------------------
\label{cap2:sec:Mecanica clasica}


Consideremos una partícula puntual de masa $m$ que se mueve en una dimensión bajo la influencia de un potencial independiente del tiempo $V(q)$. De modo que la evolución de la trayectoria de la partícula viene dada por la ecuación de movimiento de Newton, es decir,
\begin{equation*}
    m\ddot{q}=-\frac{dV}{dq}.
\end{equation*}  
Podemos describir este mismo sistema utilizando otro formalismo, encontrando una función $L$ que codifique la evolución del sistema llamada \emph{lagrangiana},
 
\begin{equation*}
    L(q,\dot{q})\equiv T-V=\frac{1}{2}m\ddot{q}^2-V(q),
\end{equation*}
por lo que podemos definir una acción $S$, tal que,
 \begin{equation*}
 S=\int_{\tau_0}^{\tau_1} d\tau L(q(\tau),\dot{q}(\tau)),
 \end{equation*}
 con la cual podemos construir las ecuaciones de movimiento por medio de un principio variacional, es decir el principio de Hamilton, en donde las trayectorias que describen el movimiento real del sistema son aquellas que minimicen la acción. Con lo cual obtenemos la llamada \emph{la ecuación de Euler-Lagrange}.

 \begin{equation*}
    \frac{d}{dt}\frac{\partial L}{\partial \dot{q}}-\frac{\partial L}{\partial q}=0.
\end{equation*} 
  
 Otra formulación equivalente se obtiene definiendo el momento canónico conjugado $p$, definiendo como:
\begin{equation*}
    p=\frac{\partial L}{\partial \dot{q}},
\end{equation*}

usando una transformación de Legendre podemos cambiar del par de variables $(q,\dot{q})$ a las variables $(q,p)$, definiendo una nueva función $H$, llamada Hamiltoniano,
\begin{equation*}
    H(q,p)\equiv p\dot{q}-L(q,\dot{q}).
\end{equation*} 
 De modo que con esta nueva formulación las ecuaciones de movimiento están dadas por las \emph{ecuaciones de Hamilton},

\begin{equation*}   
    \begin{split}        
        \dot{p}&=-\frac{\partial H}{\partial q},\\
        \dot{q}&=\frac{\partial H}{\partial p},
    \end{split}
\end{equation*} 
 
 Por último, tenemos el formalismo de los corchetes de Poisson, donde se definen para dos variables dinámicas cualquiera $A(q,p)$ y $B(q,p)$ que dependan de $q$ y $p$,
\begin{equation*}
    \lbrace A,B \rbrace=\frac{\partial A}{\partial q}\frac{\partial B}{\partial p}-\frac{\partial B}{\partial q}\frac{\partial A}{\partial p}.
\end{equation*}

Veamos que ocurre en el caso de los siguientes corchetes de Poison $\lbrace q,p \rbrace$, $\lbrace q,q \rbrace$ y $\lbrace p,p \rbrace$,

\begin{equation*}
    \begin{split}
        \lbrace q,p \rbrace&=\frac{\partial q}{\partial q}\frac{\partial p}{\partial p}-\frac{\partial p}{\partial q}\frac{\partial q}{\partial p}=1,\\
        \lbrace q,q \rbrace&=\frac{\partial q}{\partial q}\frac{\partial q}{\partial p}-\frac{\partial q}{\partial q}\frac{\partial q}{\partial p}=0,\\
        \lbrace p,p \rbrace&=\frac{\partial p}{\partial q}\frac{\partial p}{\partial p}-\frac{\partial p}{\partial q}\frac{\partial p}{\partial p}=0.\\
    \end{split}
\end{equation*} 

Los cuales son de vital importancia al pasar al formalismo clásico. Por otro lado si queremos encontrar la derivada temporal de la variable dinámica $A(q,p)$, usando las ecuaciones de Hamilton y la definición de los corchetes de Poisson, tenemos,
\begin{equation*}
    \begin{split}
        \frac{dA}{dt}&=\frac{\partial A}{\partial q}\dot{q}+\frac{\partial A}{\partial p}\dot{p}+\frac{\partial A}{\partial t},\\
        &=\frac{\partial A}{\partial q}\frac{\partial H}{\partial p}-\frac{\partial A}{\partial p}\frac{\partial H}{\partial q}+\frac{\partial A}{\partial t},\\
        &=\frac{\partial A}{\partial t}+\lbrace A,H \rbrace.
    \end{split}
\end{equation*} 
 
Los métodos de Lagrange y Hamilton brindan una elegancia y flexibilidad a la hora de describir la dinámica de un sistema.\
Todos estos resultados se pueden generalizar al caso de un número arbitrario $N$ de grados de libertad. De modo que en lugar de tener solo una coordenada $q$ y una $p$, tenemos dos conjuntos de variables dinámicas $q_{i}$ y $p_{i}$ con $i=1,2,...,N$.\

%-------------------------------------------------------------------
\section{Mecánica cuántica}
%-------------------------------------------------------------------
\label{cap3:sec:Mecanica cuantica}

El tratamiento más recurrido para pasar de un formalismo clásico a uno cuántico es la llamada \emph{cuantización de Heisenberg} o primera cuantización; la cual nos dice que tenemos que remplazar las variables dinámicas clásicas del sistema por operadores lineales que actúan sobre un espacio de Hilbert.\
Las variables dinámicas clásicas que se utilizan son $(q,p)\rightarrow(\hat{q},\hat{p})$, operadores cuánticos que no conmutan, es decir, que para cualquiera par de operadores cuánticos $\hat{A}$ y $\hat{B}$se tiene que $[\hat{A},\hat{B}]=\hat{A}\hat{B}-\hat{B}\hat{A}$ que en general es diferente de cero.\\\\
 Este objeto recibe el nombre de \emph{conmutador}, el cual esta relacionado con los corchetes de Poisson de la forma,
 
 \begin{equation*}
 \lbrace A,B \rbrace \rightarrow \frac{1}{i\hbar}[\hat{A},\hat{B}].
 \end{equation*}

Un especial de esta relación ocurre con los operadores $\hat{q},\hat{p}$,

\begin{equation*}
[\hat{A},\hat{B}]=i\hbar
\end{equation*}

Por otro lado sabemos que en la notación de Dirac podemos denotar a cualquier estado de un sistema cuántico por medio del ket $ \ket{\psi} $, además si $ \ket{\psi} $ y $ \ket{\varphi} $ son estados admisibles por el sistema, entonces también lo es su superposición,

\begin{equation*}
\alpha \ket{\psi}+\beta \ket{\varphi}=\ket{\alpha\psi+\beta\varphi}  \hspace{1cm}\forall \alpha,\beta \in \mathbb{C},
\end{equation*} 

Es decir, los estados son elementos de un espacio vectorial complejo, dotado de un producto interno entre los estados $ \ket{\psi} $ y $ \ket{\varphi} $ dado por$ \braket{\varphi}{\psi} \in \mathbb{C} $ que satisface las siguientes propiedades:

\begin{equation*}
    \begin{split}
        &\braket{\varphi}{\psi}=\braket{\psi}{\varphi}^{*}\\
        &\braket{\phi}{\alpha\psi+\beta\varphi}=\alpha\braket{\phi}{\psi}+\beta\braket{\phi}{\varphi},\\
        &\braket{\alpha\psi+\beta\varphi}{\phi}=\alpha^{*}\braket{\psi}{\phi}+\beta^{*}\braket{\varphi}{\phi},\\
        &\braket{\psi}{\psi}\geq 0 \hspace{0.5cm} \forall \ket{\psi},\\
        &\braket{\psi}{\psi}=0 \hspace{0.5cm} \text{solo si} \ket{\psi}=0,
    \end{split}
\end{equation*}

Físicamente podemos interpretar a este producto interno $ \braket{\varphi}{\psi} $ como una amplitud de probabilidad de que el sistema se encuentre en $ \ket{\varphi} $ si esta en el estado $ \ket{\psi} $ solo si los estados están normalizados, es decir $ \braket{\psi}{\psi}=\braket{\varphi}{\varphi}=1 $.

Ademas supondremos que este espacio vectorial complejo equipado con este producto interno es completo y constituye un espacio de Hilbert $\mathcal{H}$.\\

Podemos notar que los  mapeos del tipo $ \braket{\psi}{\cdot}\colon \mathcal{H}\longrightarrow \mathbb{C} $ es decir,  $\ket{\varphi}  \longmapsto \braket{\psi}{\varphi}$ forman también un espacio vectorial donde el vector dual a $ \ket{\psi} $ esta dado por $ \braket{\psi}{\cdot}\equiv\bra{\psi} $ que se conoce como bra.\\

Aterrizando esto a un sistema físico podemos denotar a $ \ket{\textbf{x}} $ como al estado de una partícula localizada en la posición $ \textbf{x} $ y sea $ \ket{\psi} $ su estado general, entonces podríamos formar el "traslape" de estos dos estados $ \braket{\textbf{x}}{\psi} $ que no es otra cosa que la amplitud de probabilidad de que la partícula con estado $ \ket{\psi} $ se encuentre en la posición $ \textbf{x} $ o lo que es lo mismo su función de onda en el espacio de posiciones. De modo que,
\begin{equation*}
\braket{\textbf{x}}{\psi}\equiv \psi(\textbf{x}).  
\end{equation*}  
Por otro lado tenemos que los estados $\{ \ket{\textbf{x}} \arrowvert \hspace{0.1cm} \textbf{x}\in \mathbb{R}^{3} \}$ nos generan una base ortogonal y completa, por lo que cumplen las siguientes propiedades,   

\begin{equation*}
    \begin{split}
	    \braket{ \textbf{x}'}{\textbf{x}}=\delta(\textbf{x}-\textbf{x}'),\\
        \int_{-\infty}^{\infty} d^{3}\textbf{x} \ket{\textbf{x}}\bra{\textbf{x}}=\mathbb{1}.
    \end{split}	
\end{equation*}

De igual manera podemos definir los estados $ \ket{\textbf{p}} $ a través de $ \braket{ \textbf{x}}{\textbf{p}}=e^{i\frac{\textbf{p}\cdot\textbf{x}}{\hbar}} $ nos brinda una nueva base para el mismo espacio de Hilbert, de modo que las relaciones de ortogonalidad y completes son, 

\begin{equation*}
    \begin{split}
        &\braket{ \textbf{p}'}{\textbf{p}}=(2\pi)^{3}\delta(\textbf{p}-\textbf{p}'),\\
        &\int_{-\infty}^{\infty} \frac{d^{3}\textbf{p}}{(2\pi)^{3}} \ket{\textbf{p}}\bra{\textbf{p}}=\mathbb{1}.
    \end{split}
\end{equation*}

Regresando al formalismo de la mecánica cuántica, cada observable físico $A$  esta asociado a un operador lineal  $\hat{A}$ tal que $ \hat{A} \colon \mathcal{H} \longrightarrow \mathcal{H} $, es decir, $ \ket{\psi}\longmapsto  \hat{A} \ket{\psi} =\ket{\text{\^{A}}\psi} $. De modo que obtenemos una ecuación de valores propios $ \text{\^{A}}\ket{\psi_{n}}=\lambda_{n}\ket{\psi_{n}} $ donde $ \lambda_{n} $ son los valores propios del operador \^{A} y $ \ket{\psi_{n}} $ sus vectores propios que constituyen una base ortogonal y completa de modo que podemos usarla para poder obtener el valor esperado del operador \^{A}, esto es,

\begin{equation*}
    \begin{split}
        \expval{\text{\^{A}}}{\psi}&=\bra{\psi}\text{\^{A}} \displaystyle\sum_{n=1} \ket{\psi_{n}}\bra{\psi_{n}}\ket{\psi},\\
        &=\displaystyle\sum_{n=1}\bra{\psi}\text{\^{A}} \ket{\psi_{n}}\bra{\psi_{n}}\ket{\psi},\\
        &=\displaystyle\sum_{n=1}\bra{\psi}\lambda_{n}\ket{\psi_{n}}\bra{\psi_{n}}\ket{\psi},\\
        &=\displaystyle\sum_{n=1}\lambda_{n}\bra{\psi_n}\ket{\psi}^{*}\bra{\psi_{n}}\ket{\psi},\\
        &=\displaystyle\sum_{n=1}\lambda_{n}\lVert\bra{\psi_{n}}\ket{\psi}\rVert^{2},
    \end{split}
\end{equation*}

Por otro lado podemos definir a los traslapes $ \bra{\psi_{n}}{\text{\^{A}}}\ket{\psi_{m}}\equiv A_{nm} $ como los elementos de matriz del operador \^{A} y lo definen por completo.\\\\
Dado un operador cualquiera $\hat{A}$ definimos su \emph{operador hermitico conjugado} $\hat{A}^{\dagger}$ tal que,

\begin{equation*}
    \bra{\varphi}\ket{\hat{A}\psi}=\bra{\hat{A}^{\dagger}\varphi}\ket{\psi} \hspace{0.5cm} \forall \ket{\varphi}, \ket{\psi}  
\end{equation*} 



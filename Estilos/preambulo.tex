%
%                       Pre�mbulo del documento
%
%           Contiene todos los paquetes necesarios para la
%                       compilaci�n del documento
%
%-------------------------------------------------------------------------
%
%Copyright 2020 David Torrez Reyes
%This file is part of BookTemplate.
%
%BookTemplate is free software: you can redistribute it and/or modify
%it under the terms of the GNU General Public License as published by
%the Free Software Foundation, either version 3 of the License, or
%(at your option) any later version.
%
%BookTemplate is distributed in the hope that it will be useful,
%but WITHOUT ANY WARRANTY; without even the implied warranty of
%MERCHANTABILITY or FITNESS FOR A PARTICULAR PURPOSE.  See the
%GNU General Public License for more details.
%
%You should have received a copy of the GNU General Public License
%along with BookTemplete.  If not, see <https://www.gnu.org/licenses/>.
%-------------------------------------------------------------------------




%                  Paquetes de idiomas y codificaci�n
%-------------------------------------------------------------------------
% El paquete inputenc modifica la codificaci�n de entrada de LaTeX.
% Permite escribir con acentos, s�mbolos inusuales, etc. Como par�metro 
% opcional ingresamos la codificaci�n deseada (latin1) ISO 8859-1.  
\usepackage[latin1]{inputenc}


% El paquete babel gestiona el idioma en el cual se desea trabajar, en
% este caso espa�ol (spanish)
\usepackage[spanish]{babel}


% El paquete fontenc gestiona la codificaci�n de salida del documento.
% EL par�metro (T1) indica la codificaci�n est�ndar de LaTeX
\usepackage[T1]{fontenc}




%                  Paquetes de gr�ficos, tablas
%-------------------------------------------------------------------------
% El entorno multicol, es de gran utilidad en los sitios donde se
% quieren poner dos figuras una al lado de la otra.
\usepackage{multicol}


% El entorno tabularx, sirve para hacer tablas con  p�rrafos sin tener que
% indicar  el ancho  de cada  columna de p�rrafo particular.
\usepackage{tabularx}

% Paquete necesario para la adici�n de figuras.
% Con el comando \graphcspath indicamos el directorio en el cual se encuentran
% las im�genes a utilizar 
\usepackage{graphicx}
\graphicspath{{Pictures/}}




%                 Paquetes de configuraci�n de paginas 
%-------------------------------------------------------------------------
% Paquete geometry, gestiona los margenes del documento.
\usepackage[top=3cm,bottom=3cm,left=3cm,right=3cm]{geometry}




%                 Paquetes para la gesti�n de colores 
%-------------------------------------------------------------------------
% EL paquete xcolor, necesario para la creaci�n y uso de colores.
\usepackage{xcolor}




%     Paquetes para la creaci�n de ecuaciones y simbolog�a matem�tica 
%-------------------------------------------------------------------------
% Paquetes necesarios para introducir simbolog�a matem�tica, ecuaciones,
% teoremas, etc.
\usepackage{amsmath}
\usepackage{amsfonts}
\usepackage{amssymb}
\usepackage{amsthm}

%
%                       Preámbulo del documento
%
%           Contiene todos los paquetes necesarios para la
%                       compilación del documento
%
%-------------------------------------------------------------------------
%
%Copyright 2020 David Torrez Reyes
%This file is part of BookTemplate.
%
%BookTemplate is free software: you can redistribute it and/or modify
%it under the terms of the GNU General Public License as published by
%the Free Software Foundation, either version 3 of the License, or
%(at your option) any later version.
%
%BookTemplate is distributed in the hope that it will be useful,
%but WITHOUT ANY WARRANTY; without even the implied warranty of
%MERCHANTABILITY or FITNESS FOR A PARTICULAR PURPOSE.  See the
%GNU General Public License for more details.
%
%You should have received a copy of the GNU General Public License
%along with BookTemplete.  If not, see <https://www.gnu.org/licenses/>.
%-------------------------------------------------------------------------




%                  Paquetes de idiomas y codificación
%-------------------------------------------------------------------------
% El paquete inputenc modifica la codificación de entrada de LaTeX.
% Permite escribir con acentos, símbolos inusuales, etc. Como parámetro 
% opcional ingresamos la codificación deseada (latin1) ISO 8859-1.  
\usepackage[utf8]{inputenc}


% El paquete babel gestiona el idioma en el cual se desea trabajar, en
% este caso español (spanish)
\usepackage[spanish]{babel}


% El paquete fontenc gestiona la codificación de salida del documento.
% EL parámetro (T1) indica la codificación estándar de LaTeX
\usepackage[T1]{fontenc}




%                  Paquetes de gráficos, tablas
%-------------------------------------------------------------------------
% El entorno multicol, es de gran utilidad en los sitios donde se
% quieren poner dos figuras una al lado de la otra.
\usepackage{multicol}


% El entorno tabularx, sirve para hacer tablas con  párrafos sin tener que
% indicar  el ancho  de cada  columna de párrafo particular.
\usepackage{tabularx}

% Paquete necesario para la adición de figuras.
% Con el comando \graphcspath indicamos el directorio en el cual se encuentran
% las imágenes a utilizar 
\usepackage{graphicx}
\graphicspath{{Imagenes/}}




%                 Paquetes de configuración de paginas 
%-------------------------------------------------------------------------
% Paquete geometry, gestiona los margenes del documento.
\usepackage[top=3cm,bottom=3cm,left=3cm,right=3cm]{geometry}
\usepackage{lipsum}



%                 Paquetes para la gestión de colores 
%-------------------------------------------------------------------------
% EL paquete xcolor, necesario para la creación y uso de colores.
\usepackage{xcolor}




%     Paquetes para la creación de ecuaciones y simbología matemática 
%-------------------------------------------------------------------------
% Paquetes necesarios para introducir simbología matemática, ecuaciones,
% teoremas, etc.
\usepackage{amsmath}
\usepackage{amsfonts}
\usepackage{amssymb}
\usepackage{amsthm}

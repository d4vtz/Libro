%
%                           	Tablas de Contenido
%
%-------------------------------------------------------------------------
%
%Copyright 2020 David Torrez Reyes
%This file is part of BookTemplate.
%
%BookTemplate is free software: you can redistribute it and/or modify
%it under the terms of the GNU General Public License as published by
%the Free Software Foundation, either version 3 of the License, or
%(at your option) any later version.
%
%BookTemplate is distributed in the hope that it will be useful,                 
%but WITHOUT ANY WARRANTY; without even the implied warranty of
%MERCHANTABILITY or FITNESS FOR A PARTICULAR PURPOSE.  See the
%GNU General Public License for more details.
%
%You should have received a copy of the GNU General Public License
%along with BookTemplete.  If not, see <https://www.gnu.org/licenses/>.
%-------------------------------------------------------------------------




% En este archivo se generaran las tablas de contenido como puede ser,
% lista de capítulos, secciones, figuras, ejercicios, etc.  
%                           Configuración
%-------------------------------------------------------------------------

%  Con este comando manejamos el nivel de items de la tabla de contenidos
%  1. Capítulo
%       1.1. Sección
\setcounter{tocdepth}{2}

% Le  pedimos  que nos  numere  todos  los  items hasta  el  nivel
% \subsubsection en el cuerpo del documento.
\setcounter{secnumdepth}{3}

% Creamos los diferentes índices.
% Lo primero un  poco de trabajo en los marcadores  del PDF. No quiero
% que  salga una  entrada  por cada  índice  a nivel  0...  si no  que
% aparezca un marcador "Índices", que  tenga dentro los otros tipos de
% índices.  Total, que creamos el marcador "Índices".
% Antes de  la creación  de los índices,  se añaden los  marcadores de
% nivel 1.

%\ifpdf
%   \pdfbookmark{Índices}{indices}
%\fi



%               Tablas de Contenidos
%....................................................

% No se que haga... Manejar los marcadores del pdf
\ifpdf
   \pdfbookmark[1]{Tabla de contenidos}{tabla de contenidos}
\fi

% Generamos la cabecera del Índice
\cabecera{Índice}
% Creamos la tabla de contenidos
\tableofcontents
% Nueva pagina
\newpage 



%               Índice de Figuras
%....................................................

% No se que haga... Manejar los marcadores del pdf
\ifpdf
   \pdfbookmark[1]{Índice de figuras}{indice de figuras}
\fi

% Creamos el indice de figuras
\listoffigures
% Nueva pagina
\newpage





%               Índice de Tablas
%....................................................

% No se que haga... Manejar los marcadores del pdf
\ifpdf
   \pdfbookmark[1]{Índice de tablas}{indice de tablas}
\fi

% Creamos el indice de figuras
\listoftables
% Nueva pagina
\newpage
\cleardoublepage
\restauraCabecera
\pagenumbering{arabic}
